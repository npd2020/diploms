\documentclass[a4paper, 14pt]{article}
\usepackage{comment}

\usepackage{setspace}
\usepackage{indentfirst}
%% Language and font encodings
\usepackage{extsizes}
\usepackage[english, russian, ukrainian]{babel}
\usepackage[utf8x]{inputenc}
\usepackage[T1]{fontenc}
\linespread{1.6}
%% Sets page size and margins
\usepackage[a4paper,top=2cm,bottom=2cm,left=3cm,right=2cm,marginparwidth=1.75cm]{geometry}

%\usepackage{fontspec}

%% Useful packages
\usepackage{authblk}
\usepackage{amsmath}
\usepackage{graphicx}
\usepackage[colorinlistoftodos]{todonotes}
\usepackage[colorlinks=true, allcolors=black]{hyperref}
\usepackage{tikz}
%\usepackage{subfigure}
\usepackage[lofdepth,lotdepth]{subfig}
\usepackage{float}
\usepackage{multirow}
\usepackage{hhline}
\usepackage{lineno}

%\linenumbers
\renewcommand{\thefigure}{\thesection.\arabic{figure}}

\numberwithin{equation}{section}
\numberwithin{table}{section}

\title{}


\author[1]{V. Haponov}
\author[2]{R. Yermolenko}
\affil[1]{Taras Shevchenko National University of Kiev, Kiev, Ukraine}
\affil[2]{}

\setcounter{Maxaffil}{0}
\renewcommand\Affilfont{\itshape\normalsize}
\renewcommand{\arraystretch}{1.5} %% increase table row spacing
%\renewcommand{\tabcolsep}{1cm}

\date{}
\begin{document}
	%%%%%%%%%%%%%%%%%%%%%%%%%%%%%%%%%%%%%%%%%%%%%%%%%%%%%%%%%%%%%%%%%%%%%%%%%%%%%%%%%%%%%%%%%%%%%%%%%%%%%%%%%%%%%%%%%%%%%%%%%%%%%%%%%%%%%%%%%%%%%%%%%%%%%%%%%%%%%%%%%%%%%%%%%%%%%%%%%%%%%%%%%%%%%%%%%%%%%%%%%%%TITLE_PAGE%%%%%%%%%%%%%%%%%%%%%%%%%%%%%%%%%%%%%%%%%%%%%%%%%%%%%%%%%%%%%%%%%%%%%%%%%%%%%%%%%%%%%%%%%%%%%%%%%%%%%%%%%%%%%%%%%%%%%%%%%%%%%%%%%%%%%%%%%%%%%%%%%%%%%%%%%%%%%%%%%%%%%%%%%%%%%%%%%%%%%%%%%%%%%%%%%%%%%%%%%%%%%%%%%%%%
	\begin{titlepage}
		\renewcommand{\baselinestretch}{1.0}
		\selectlanguage{ukrainian}
		\begin{center}
			КИЇВСЬКИЙ НАЦІОНАЛЬНИЙ УНІВЕРСИТЕТ ІМЕНІ ТАРАСА ШЕВЧЕНКА\\Фізичний факультет\\Кафедра ядерної фізики
		\end{center}
		\vspace*{1.5cm}
		{}\hfill\mbox{На правах рукопису}
		
		\vspace*{3cm}
		\begin{center} {\bf Дослідження можливості застосування нейтронно-активаційного аналізу для пошуку корисних копалин в глибинах океану
			}
		\end{center}
		\medskip
		\vspace*{0.7cm}
		\begin{flushleft}
			\parbox{12cm}{
				\textbf{Галузь знань:} 10 <<Природничі науки>>
				
				\textbf{Освітня програма} - Фізика
				
				\textbf{Спеціальність} - 104 <<Фізика та астрономія>>
				
				\textbf{Спеціалізація} Ядерна енергетика
			}
		\end{flushleft}
		\renewcommand{\baselinestretch}{1.5}
		\vspace*{1cm}
		{}\hfill\hspace{7.5cm}\parbox{9cm}{\textbf{Кваліфікаційна робота бакалавра}\\
			студента 4 курсу\\ Гапонова Валентина Вікторовича \\ \\ 
			\textbf{Науковий керівник} \\ канд. ф.-м. наук\\ Єрмоленко Руслан Вікторович}
		\bigskip
		
		
		\vfill
		{\small \noindent
			Робота заслухана на засіданні кафедри ядерної фізики та рекомендована до захисту на ЕК, протокол , протокол № \underline{\hspace{1.0cm}}  від <<\underline{\hspace{1.0cm}}>> \underline{\hspace{3.5cm}}2020 р.\\[0.4cm]
			Завідувач кафедри \hspace{9 cm} Каденко І. М.}
		%\bigskip
		\vfill
		\begin{center} Київ, 2020 \end{center}
		
	\end{titlepage}
	
	\begin{titlepage}
		\renewcommand{\baselinestretch}{1.0}
		\selectlanguage{ukrainian}
		\newcommand{\ul}[1]{\rule{#1}{0.1pt}}
		\begin{spacing}{1.8}
			\vspace*{4.5cm}
			{\center
				{\bf ВИТЯГ}\\
				з протоколу № \ul{2.4cm}\\
				засідання Екзаменаційної комісії\\[2cm]}
			{\noindent
				Визнати, що студент \ul{7.2cm} виконав та захистив кваліфікаційну роботу бакалавра з оцінкою \ul{7.2cm} .\\[1cm]}
			{\flushright
				Голова ЕК \ul{7.8cm}\\
				<<\ul{1cm}>> \ul{4cm} 2020 р.\\}
		\end{spacing}
	\end{titlepage}
	

	
\end{document}