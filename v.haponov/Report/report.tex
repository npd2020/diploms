\documentclass[a4paper, 14pt]{article}
\usepackage{comment}

\usepackage{setspace}
\usepackage{indentfirst}
%% Language and font encodings
\usepackage{extsizes}
\usepackage[english, russian, ukrainian]{babel}
\usepackage[utf8x]{inputenc}
\usepackage[T1]{fontenc}
\linespread{1.6}
%% Sets page size and margins
\usepackage[a4paper,top=2cm,bottom=2cm,left=3cm,right=2cm,marginparwidth=1.75cm]{geometry}

%\usepackage{fontspec}

%% Useful packages
\usepackage{authblk}
\usepackage{amsmath}
\usepackage{graphicx}
\usepackage[colorinlistoftodos]{todonotes}
\usepackage[colorlinks=true, allcolors=black]{hyperref}
\usepackage{tikz}
%\usepackage{subfigure}
\usepackage[lofdepth,lotdepth]{subfig}
\usepackage{float}
\usepackage{multirow}
\usepackage{hhline}
\usepackage{lineno}

%\linenumbers
\renewcommand{\thefigure}{\thesection.\arabic{figure}}

\numberwithin{equation}{section}
\numberwithin{table}{section}

\title{}


\author[1]{V. Haponov}
\author[2]{R. Yermolenko}
\affil[1]{Taras Shevchenko National University of Kiev, Kiev, Ukraine}
\affil[2]{}

\setcounter{Maxaffil}{0}
\renewcommand\Affilfont{\itshape\normalsize}
\renewcommand{\arraystretch}{1.5} %% increase table row spacing
%\renewcommand{\tabcolsep}{1cm}

\date{}
\begin{document}
	
%content

\selectlanguage{ukrainian}
\newpage
\tableofcontents
\newpage
\pagestyle{plain}
\setcounter{page}{2}
	
%end 

\newpage
\section{Глава 1 \\}
\setcounter{figure}{0} 

\subsection{$QGSP\_BERT$}
	$QGSP\_BERT\_HP$ - ця фізична модель входить в перелік стандартних фізичних моделей розрахункового пакету Geant4
	Базується на каскадній моделі Бертіні та враховує реакції для нейтронів менше ніж 20 МеВ. Для валідації данної моделі необхідне виконання наступних умов $\frac{\lambda_B}{\nu} \ll \tau_c \ll \Delta{t}$, $\lambda_B$ - хвиля де-Бролля для налітаючої частинки, $\nu$ - швидкість налітаючої частинки, $\Delta{t}$ - час між зіткненнями. Та модель яка лягла в основу коду Geant4 була протестована на частинках з енергіями від 100 МеВ до 3 ГеВ
	
\newpage
\section{Глава 2 \\}
\setcounter{figure}{0}

\subsection{Опис детектора}
	
	Для моделювання чутливого об'єму був обраний надчистий германій, з діаметром 60.6 міліметрів, та довжиною 56.7 міліметрів. Рис. ~\ref{ris:s_detector_volume} \\
	
	\begin{figure}[hbt!]
		%\vspace{-10pt}
		\centering \includegraphics[width=0.7\textwidth]{images/sDetector158cm3.png}
		\caption{Форма чутливого об'єму} 
		\label{ris:s_detector_volume}	
	\end{figure} 

	Детектор буде розміщенний поряд з джерелом нейтронів високих енергій, 14.5 МеВ. Тому детектор був розміщений у трьох шаровий захист. Рис. ~\ref{ris:s_detector_P}
	
	\begin{figure}[hbt!]
		%\vspace{-10pt}
		\centering \includegraphics[width=0.7\textwidth]{images/dectorPrt.png}
		\caption{Захист детектора, Al - зелений товщина 2 см., B - жовтий товщина 5 см., Pb - червоний товщина 1 см. Блакитний шар повітря} 
		\label{ris:s_detector_P}	
	\end{figure} 

	В захисті використовується Бор для поглинання теплових нейтронів, так як вся детекторна система буде знаходитися під водою, то нейтрони від джерела будуть втрачати енергію при пружному розсіянні на водню. 
	
	Опис реакцій на захисті -- та сповільнювачі \\
	Опис вторинного альфа випроміннення від Бор --
	Fano factor Ge = 0.13 
	Ширина забороненої зони 0,67 Т = 300 К
	
\subsection{Опис коду моделі}

	\begin{figure}[hbt!]
		%\vspace{-10pt}
		\centering \includegraphics[width=1\textwidth]{res/classDiagram.pdf}
		\caption{Діаграма класів коду моделі} 
		\label{ris:s_classDiagram}	
	\end{figure} 

	Цілью було написати максимально зручний код для набору спектрів за різних умов та на різних мішеннях, тому були створені абстрактні класси для створення геометричних об'єктів. Для зручності створення матеріалів були створенні структури. 
	Та для пришвидшення роботи були всі можливі константи ініціалізувалися на етапі компіляції. Для полегшення контролю над пам'яттю використовувалися розумні вказівники С++ 11 стандарту.
	
\newpage 
\section{Глава 3}
\setcounter{figure}{0}

\subsection{Валідація моделі}

	Для підтвердження можливості проведення наборів на моїй моделі був набраний спектор для Гірчичного газу. Рис ~\ref{ris:MustBackAllLogSm}. Та порівняний з отриманим спектром в статьї <link to source ?help> 	
	\begin{figure}[hbt!]
		%\vspace{-10pt}
		\centering \includegraphics[width=1\textwidth]{res/smMustFonAll.pdf}
		\caption{Червоним - представлений спектор для Гірчичного газу. Синім - фону} 
		\label{ris:MustBackAllLogSm}	
	\end{figure} 

	У висновках до проведеної роботи в данної статьї було вказано, що най більш вираженими і читкіми були лінії Cl 7,42 7,80 8,58 МеВ, також було вказано ряд ліній які їм не вдалося валідувати (перефразувати "лінії") 
	На Рис. ~\ref{ris:MustFon78} - зоображені піки 7.42 та 7.80 МеВ
	Рис.  ~\ref{ris:MustFon89} - 8.58 МеВ	
	\begin{figure}[hbt!]
		%\vspace{-10pt}
		\centering \includegraphics[width=1\textwidth]{res/mustFon78.pdf}
		\caption{Червоним - представлений спектор для Гірчичного газу. Синім - фону} 
		\label{ris:MustFon78}	
	\end{figure} 	
	\begin{figure}[hbt!]
		%\vspace{-10pt}
		\centering \includegraphics[width=1\textwidth]{res/mustFon89.pdf}
		\caption{Червоним - представлений спектор для Гірчичного газу. Синім - фону} 
		\label{ris:MustFon89}	
	\end{figure} 
	
	

\newpage	
%literature
\addcontentsline{toc}{section}{Література}
\begin{thebibliography}{}
	
%	\comment{кометар: це джерело звідки були взяті розміри детектора}  
	\bibitem{1} \textit{R.M. Keyser and T.R. Twomey} - Extended Source Sensitivity and Resolution Comparisons of Several HPGe Detector Types with Low-energy Capabilities \\
	\href{https://www.ortec-online.com/-/media/ametekortec/technical%20papers/high%20purity%20germanium%20detector%20applications%20and%20technology%20developements/extended-source-sensitivity-resolution-comparisons-several-hpge-detector-types-low-energy-capabilities.pdf?la=en}{ HPGe Detector Types}
		
	\bibitem{1} \textit{Aatos Heikkinen, Nikita Stepanov Helsinki Institute of Physics, P.O. Box 64, FIN-00014 University of Helsinki, Finland Johannes Peter Wellisch CERN, Geneva, Switzerland} - Bertini intra-nuclear cascade implementation in Geant4
	\href{https://www.slac.stanford.edu/econf/C0303241/proc/papers/MOMT008.PDF}{link}
	
\end{thebibliography}

\end{document}
